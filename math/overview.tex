\documentclass[a4paper,12pt]{article}
\setlength{\oddsidemargin}{0in}
\setlength{\evensidemargin}{0in}
\setlength{\textwidth}{6.5in}
\setlength{\textheight}{9.5in}
\setlength{\footskip}{.5in}

\setlength{\voffset}{-.5in}
\setlength{\topmargin}{0in}
\setlength{\headheight}{.5in}
\setlength{\headsep}{12pt}

%\usepackage{savetrees}
\usepackage{rotating}
\usepackage{algorithm}
\usepackage{algorithmic}
\usepackage{amsmath}
\usepackage{amssymb}
\usepackage{fancyhdr,lastpage}
\usepackage{enumerate}
\usepackage{graphicx}
\usepackage{epstopdf}
\usepackage[round]{natbib}
\usepackage{rotating}
\usepackage{mathrsfs}
\usepackage{tipa}
\usepackage{overpic}

\newcommand{\pd}[2]{\frac{\partial #1}{\partial #2}}
\newcommand{\td}[2]{\frac{d #1}{d #2}}
\newcommand{\pdd}[2]{\frac{\partial^2 #1}{\partial #2 \partial #2}}
\newcommand{\expv}[1]{\left \langle #1 \right \rangle}
\newcommand{\tensort}[1]{\underline{\underline{#1}}}
\newcommand{\erf}[1]{\mbox{erf}{\left( {#1}\right)}}
\newcommand{\erfc}[1]{\mbox{erfc}{\left( {#1}\right)}}
\newcommand{\ands}{ \ \ \ \mbox{ and } \ \ \ }
\newcommand{\spword}[1]{ \ \ \ \mbox{ {#1} } \ \ \ }

\begin{document}
Pieces:
\begin{enumerate}
        \item Integrate the kernels against the Legendre polynomials exactly.
        \item Use the rules from Diligenti 1997 in 2D. In 3D, I will need to develop new double integral product rules.
        \item Use the generalized vandermonde matrix to convert back and forth between Legendre Polynomials and various representations as nodal Lagrange basis functions. 
        \item Conversion between nodal and Legendre form can be done in $O(p^2)$ by multiplying by the generalized Vandermonde or its inverse.
        \item Integrals of derivatives of Legendre Polynomials against a kernel can easily be rewritten into a recurrence.
        \item I think both the original basis and the derivative of the original basis can be represented in nodal form. To convert integrals of the nodal 
        \item In 3D, the $p^2 \times p^2$ generalized Vandermonde matrix can be written as the Kronecker product of two $p \times p$ generalized Vandermonde matrices. 
\end{enumerate}

\end{document}
